\documentclass[]{article}
\usepackage[T1]{fontenc}
\usepackage{lmodern}
\usepackage{amssymb,amsmath}
\usepackage{ifxetex,ifluatex}
\usepackage{fixltx2e} % provides \textsubscript
% use upquote if available, for straight quotes in verbatim environments
\IfFileExists{upquote.sty}{\usepackage{upquote}}{}
\ifnum 0\ifxetex 1\fi\ifluatex 1\fi=0 % if pdftex
  \usepackage[utf8]{inputenc}
\else % if luatex or xelatex
  \ifxetex
    \usepackage{mathspec}
    \usepackage{xltxtra,xunicode}
  \else
    \usepackage{fontspec}
  \fi
  \defaultfontfeatures{Mapping=tex-text,Scale=MatchLowercase}
  \newcommand{\euro}{€}
\fi
% use microtype if available
\IfFileExists{microtype.sty}{\usepackage{microtype}}{}
\usepackage[margin=0.8in]{geometry}
\ifxetex
  \usepackage[setpagesize=false, % page size defined by xetex
              unicode=false, % unicode breaks when used with xetex
              xetex]{hyperref}
\else
  \usepackage[unicode=true]{hyperref}
\fi
\hypersetup{breaklinks=true,
            bookmarks=true,
            pdfauthor={Sriram V (CS11B058)},
            pdftitle={CS4110 -- Assignment 1},
            colorlinks=true,
            citecolor=blue,
            urlcolor=blue,
            linkcolor=magenta,
            pdfborder={0 0 0}}
\urlstyle{same}  % don't use monospace font for urls
\setlength{\parindent}{0pt}
\setlength{\parskip}{6pt plus 2pt minus 1pt}
\setlength{\emergencystretch}{3em}  % prevent overfull lines
\setcounter{secnumdepth}{0}

\title{CS4110 -- Assignment 1}
\author{Sriram V (CS11B058)}
\date{August 17, 2014}

\begin{document}
\maketitle

\subsection{Answer 1}\label{answer-1}

\begin{itemize}
\itemsep1pt\parskip0pt\parsep0pt
\item
  DVDs rely on a red laser to read and write data, whereas Blu-Ray uses
  a blue-violet laser instead. The benefit of using a blue-violet laser
  (405nm) is that it has a shorter wavelength than a red laser (650nm),
  which makes it possible to focus the laser spot with even greater
  precision.\\This allows data to be packed more tightly and stored in
  less space, thus making it possible to fit more data on the disc even
  though it is the size of a DVD. This together with the change of
  numerical aperture to 0.85 is what enables Blu-ray discs to hold
  25GB/50GB.
\item
  The Blu-ray disc overcomes DVD-reading issues by placing the data on
  top of a 1.1-mm-thick polycarbonate layer. Having the data on top
  prevents birefringence and therefore prevents readability problems.
\item
  Since the recording layer sits closer to the objective lens of the
  reading mechanism, the problem of disc tilt is virtually eliminated.
\end{itemize}

\subsection{Answer 2}\label{answer-2}

\begin{itemize}
\itemsep1pt\parskip0pt\parsep0pt
\item
  Bitcoin is a software-based online payment system described by Satoshi
  Nakamoto in 2008 and introduced as open-source software in 2009.
  Payments are recorded in a public ledger using its own unit of
  account, which is also called bitcoin.
\item
  As payments work peer-to-peer without a central repository or single
  administrator, it is a decentralized virtual currency.
\item
  Although its status as a currency is disputed, it is commonly called a
  cryptocurrency.
\item
  Bitcoins are mined by using computing power to verify and record
  payments into the public ledger, the block chain.
\item
  Transactions are verified by digital signatures and hence cannot be
  counterfeited, but the public ledger can be used to link transactions
  to individuals and companies.
\item
  A valid payment message from an address must contain the associated
  public key and a digital signature proving possession of the
  associated private key. Because anyone with a private key can spend
  all of the bitcoins associated with the corresponding address,
  protection of private keys is quite important.
\end{itemize}

\subsection{Answer 3}\label{answer-3}

The various virtualization options in the Linux kernel and userspace
include:

\begin{itemize}
\itemsep1pt\parskip0pt\parsep0pt
\item
  \textbf{Hardware-assisted virtualization:} Also known as accelerated
  virtualization, this approach enables efficient full virtualization
  using hardware capabilities, mainly from the host processors. The
  unmodified gues OS executes in complete isolation in the simulated
  hardware environment.
\item
  \textbf{Paravirtualization:} It presents a software interface, or
  hooks to allow the guest OS to request and acknowledge tasks, which
  would otherwise be executed virtually, leading to bad performance.
  This requires the guest OS to be explicitly ported for the para-API.
\item
  \textbf{Coopvirt (Cooperative Virtualization:} This is a hybrid of the
  above two methods, that takes advantage of technologies like Intel
  VT-x or AMD-V, as well as requiring the guest OS to interact only by
  means of the para-API.
\item
  \textbf{LXC (LinuX Containers):} It is an operating system--level
  virtualization method for running multiple isolated Linux systems
  (containers) on a single control host. LXC combines cgroups and
  namespace support to provide an isolated environment for applications.
\end{itemize}

\subsection{Answer 4}\label{answer-4}

\begin{itemize}
\itemsep1pt\parskip0pt\parsep0pt
\item
  USB 3.0 is 10 times faster than USB 2.0. It is often referred to as
  Super Speed or SS, with a transfer speed of 4.8Gbps as compared to the
  High Speed (HS) USB 2.0's speeds of 480Mbps.
\item
  The signalling method is also asychronous in USB 3.0 ie., it can send
  and receive data simultaneously (full duplex) as opposed to USB 2.0's
  half duplex polling mechanism.
\item
  USB 3.0 is also more power efficient, has 9 wires (2.0 has 4) within
  the cable and has blue Standard-A connectors (USB 2.0 has grey
  connectors). It is backwards compatible with USB 2.0
\item
  USB 3.1 offers higher speeds of 10 Gbps (SuperSpeed+), putting it on
  par with Thunderbolt Gen 1. It too is backward compatible with USB 3.0
  and 2.0.
\item
  Further, it allows devices with large energy demands request higher
  currents and supply voltages from compliant hosts.
\item
  The encoding used in 3.1 is 128b132b which is more bandwidth efficient
  compared to USB 3.0's 8b10b scheme.
\end{itemize}

\subsection{Answer 5}\label{answer-5}

\begin{itemize}
\itemsep1pt\parskip0pt\parsep0pt
\item
  Near field communication (NFC) is a set of standards for smart devices
  to establish radio communication with each other by bringing them into
  proximity (a few cm). They are based on exising RFID standards (uses
  the 13.56 MHz band). NFC uses magnetic induction between two loop
  antennas.
\item
  NFC applications can be split into four basic categories:

  \begin{itemize}
  \itemsep1pt\parskip0pt\parsep0pt
  \item
    Touch and Go: Transport ticketing, data capture applications
    (picking up a URL from a smart label on a poster).
  \item
    Touch and Confirm: Mobile payments, instead of confirming
    interactions by passwords or card-swiping)
  \item
    Touch and Connect: Linking two NFC devices to enable P2P transfer of
    data. For applications that require greater bandwidth such as games,
    this generally works by piggy-backing on a technology with greater
    bandwidth, such as bluetooth.
  \item
    Touch and Explore: NFC devices may offer multiple functions. Users
    can explore a device's capabilities to find out which services are
    offered.
  \end{itemize}
\end{itemize}

\subsection{Answer 6}\label{answer-6}

\begin{itemize}
\itemsep1pt\parskip0pt\parsep0pt
\item
  The Oculus Rift is a virtual reality head-mounted display, that is
  currently being developed by Oculus VR, which was recently acquired by
  Facebook. The Development Kit 2 was released in July 2014, and
  features higher refresh rates, head positional tracking and the higher
  resolution screen of the Note 3.
\item
  The Rift completely covers the eyes, thus shutting off the user from
  the outside world.
\item
  It works by projecting slightly shifted and distored images (with an
  FoV of 90-110 degrees) for each eye, and using a lens to adjust this
  picture (using the pincushion effect), thereby producing the effect of
  stereoscopic 3D.
\item
  Apart from this, it also uses internal and external tracking devices
  to monitor the head position along six axes as well as track eyeball
  movement, to create corresponding in-game movements within 30ms.
\end{itemize}

\subsection{Answer 7}\label{answer-7}

\begin{itemize}
\itemsep1pt\parskip0pt\parsep0pt
\item
  Thunderbolt is a technology similar to USB in that it allows
  connecting external peripherals to a computer. However, it combines
  PCIe and DisplayPort into one serial signal, together with a DC
  connection. The technology was co-developed by Apple and Intel, with
  an architecture different from that of the USB.
\item
  The first-generation Thunderbolt gave speeds of up to 10 Gbps,
  comparable to USB 3.1. Thunderbolt 2, that debuted in the 2013 Macbook
  Pro doubles this transfer rate by making use of channel aggregation.
\item
  Thus performance-wise Thunderbolt is much, much better than USB.
  However, it is currently not as widely available as USB ports. Also,
  the USB is practically free since it comes baked into chipsets from
  both AMD and Intel. Thunderbolt however, is very expensive.
\item
  An important issue to be noted is that Thunderbolt makes systems
  vulnerable to DMA attacks as it extends the PCIe bus, thereby
  providing low-level access.
\end{itemize}

\subsection{Answer 8}\label{answer-8}

\begin{itemize}
\itemsep1pt\parskip0pt\parsep0pt
\item
  \texttt{systemd} is a system management daemon designed for Linux and
  programmed exclusively for the Linux API. It is the first process to
  execute in user space during startup, hence serving as the root of the
  user space's process tree.
\item
  It is a replacement for the old script-based SysVInit (or
  \texttt{init}), and has a very different design that integrates a lot
  of modules to allow for faster boot times.
\item
  It makes use of a lot of parallelization, starting some daemons
  simultaneously.
\item
  \texttt{systemd} supports DBus and sockets, so it can be easily
  controlled. Syntax is simpler too.
\item
  As it is an init binary, it is more aware about processes and is thus
  better at logging than say, \texttt{syslog}.
\item
  People are unhappy with \texttt{systemd} for various reasons
  including:

  \begin{itemize}
  \itemsep1pt\parskip0pt\parsep0pt
  \item
    Attitude of the key devs towards users and bug reports.
  \item
    The fact that core technologies like GNOME are becoming dependent on
    it and that it also integrates other software such as \texttt{dbus}
    and \texttt{udev}, thus reducing choice of other init daemons.
  \item
    Not POSIX compliant. BSD users aren't happy.
  \item
    Everything is in one package, which is against the UNIX philosophy
    of \emph{``do one task, and do it well.''}
  \end{itemize}
\end{itemize}

\subsection{Answer 9}\label{answer-9}

\begin{itemize}
\itemsep1pt\parskip0pt\parsep0pt
\item
  Multiprotocol Label Switching (MPLS) is a mechanism in
  high-performance networks that directs data from one node to the next
  based on short path labels rather than long network addresses,
  avoiding complex lookups in a routing table. The labels identify
  virtual links (paths) between distant nodes rather than endpoints.
  MPLS can encapsulate packets of various network protocols.
\item
  Service providers are switching their core networks to MPLS because it
  provides:

  \begin{itemize}
  \itemsep1pt\parskip0pt\parsep0pt
  \item
    Improved up-time
  \item
    Improved bandwidth utilization
  \item
    Reduced network congestion
  \item
    Quality of Service (QoS): Higher standards such as reliability,
    speed, and voice quality.
  \item
    Ability to assign priorities to packets based on labels
  \end{itemize}
\end{itemize}

\subsection{Answer 10}\label{answer-10}

\begin{itemize}
\itemsep1pt\parskip0pt\parsep0pt
\item
  Wireless charging essentially works on the principle of
  electromagnetic induction, making use of an electromagnetic field to
  transfer energy via an inductive coupling to an electrical device,
  which then uses this energy to charge batteries and/or run the device.
\item
  Induction chargers typically use an induction coil to create an
  alternating electromagnetic field from within a charging base station,
  and a second induction coil in the portable device takes power from
  the electromagnetic field and converts it back into electrical current
  to charge the battery. The two induction coils in proximity combine to
  form an electrical transformer.
\item
  Advantages include protected connections and durability (which are
  especially of great use for implanted devices, in the medical field),
  while prime disadvantages are lower efficieny, heat generation, and
  slower charging. It is also more expensive than conventional charging
  methods.
\end{itemize}

\subsection{Answer 11}\label{answer-11}

\begin{itemize}
\itemsep1pt\parskip0pt\parsep0pt
\item
  Li-Fi is a wireless optical networking technology that uses
  light-emitting diodes (LEDs) for data transmission.
\item
  LiFi is designed to use LED light bulbs similar to those currently in
  use in many energy-conscious homes and offices. However, LiFi bulbs
  are outfitted with a chip that modulates the light imperceptibly for
  optical data transmission. LiFi data is transmitted by the LED bulbs
  and received by photoreceptors.
\item
  Some commercial kits providing speeds of 150 Mbps are available,
  although researchers have enabled 10 Gbps speeds with stronger LEDs
  and modified technology.
\item
  Benefits of LiFi:

  \begin{itemize}
  \itemsep1pt\parskip0pt\parsep0pt
  \item
    Higher speeds than Wi-Fi.
  \item
    10000 times the frequency spectrum of radio.
  \item
    More secure because data cannot be intercepted without a clear line
    of sight.
  \item
    Prevents piggybacking.
  \item
    Eliminates neighboring network interference.
  \item
    Unimpeded by radio interference.
  \item
    Does not create interference in sensitive electronics, making it
    better for use in environments like hospitals and aircraft.
  \end{itemize}
\item
  Drawbacks include the need for a clear line of sight, difficulties
  with mobility and the requirement that lights stay on for operation.
\end{itemize}

\subsection{Answer 12}\label{answer-12}

\begin{itemize}
\itemsep1pt\parskip0pt\parsep0pt
\item
  The X Window System, or X11, is a network-transparent windowing system
  for bitmap displays. It acts as the interface between inputs and
  outputs.
\item
  X11 makes use of a client-server model. The server (on the host)
  communicates with the client programs via TCP port 6000. Thus, it is
  not necessary for the client programs to be running on the host
  machine. One of the fundamental design principles of X is to
  \emph{``not serve all the world's needs; rather make the system
  extensible.''}
\item
  From a developer's perspective, Wayland provides a smaller codebase
  that follows a good programming model and has a better API than the X
  server, which has become bloated by wrapping it in more and more
  extensions and plugins.
\item
  The X server also has tearing issues due to no media coherence, which
  is taken care of by a separate compositor. Wayland merges the server
  with the compositor.
\item
  Wayland also delegates all rendering responsibilities to the clients.
\end{itemize}

\subsection{Answer 13}\label{answer-13}

\begin{itemize}
\itemsep1pt\parskip0pt\parsep0pt
\item
  D-Bus is a free, open source inter-process communication (IPC) system,
  allowing multiple, concurrently running processes to communicate with
  each other. It provides commmunication between:

  \begin{itemize}
  \itemsep1pt\parskip0pt\parsep0pt
  \item
    desktop apps in the same desktop session
  \item
    desktop session and the OS, including the kernel and any system
    daemons or processes
  \end{itemize}
\item
  D-Bus also makes use of a bus topology thereby allowing more than one
  process to receive a message. It is thus something similar to Unix
  sockets, but has low overhead due to a binary protocol.
\item
  It helps us in various ways, say when we receive a phone call, D-Bus
  can send a message to the volume control to mute the audio that might
  be playing from the speakers and notify us of the call.
\end{itemize}

\subsection{Answer 14}\label{answer-14}

\begin{itemize}
\itemsep1pt\parskip0pt\parsep0pt
\item
  Phase-change memory (PCM or PRAM) is a type of non-volatile
  random-access memory. PRAMs exploit the unique behaviour of
  chalcogenide glass, namely that its optical and electrical properties
  can be modified by switching it between an amporphous and crystalline
  state, thereby allowing the storage of information.
\item
  Advantages:

  \begin{itemize}
  \itemsep1pt\parskip0pt\parsep0pt
  \item
    Fast switching time and inherent scalability: This offers much
    higher performance in applications where writing quickly is
    important.
  \item
    Single bits may be changed without requiring to erase an entire
    block first.
  \item
    PRAMs degrade much more slowly than Flash, which degrades with each
    burst of voltage across the cell.
  \end{itemize}
\item
  Drawbacks:

  \begin{itemize}
  \itemsep1pt\parskip0pt\parsep0pt
  \item
    High programming current density is needed (\textgreater{}107 A/cm²,
    compared to 105-106 A/cm² for a typical transistor or diode).
  \item
    The contact between the hot phase-change region and the adjacent
    dielectric is another fundamental concern.
  \item
    The dielectric may begin to leak current at higher temperature, or
    may lose adhesion when expanding at a different rate from the
    phase-change material.
  \item
    The resistance of the amorphous state slowly increases according to
    a power law, and could jeopardize standard two-state operation if
    the threshold voltage increases beyond the design value.
  \end{itemize}
\end{itemize}

\subsection{Answer 15}\label{answer-15}

\begin{itemize}
\itemsep1pt\parskip0pt\parsep0pt
\item
  x86:

  \begin{itemize}
  \itemsep1pt\parskip0pt\parsep0pt
  \item
    The x86 architecture is a variable instruction length, primarily
    ``CISC'' design with emphasis on backward compatibility.
  \item
    Byte-addressing is enabled and words are stored in memory with
    little-endian byte order. It is a 32-bit processor.
  \item
    The current x86 architecture has been influenced by designs from
    both Intel and AMD.
  \item
    Examples include the Intel Pentium family, as well as the Atom
    processors.
  \end{itemize}
\item
  x86-64/AMD64/x64:

  \begin{itemize}
  \itemsep1pt\parskip0pt\parsep0pt
  \item
    It is the 64-bit version of x86. It has a CISC architecture as well.
  \item
    It supports vastly larger amounts of virtual memory and physical
    memory than is possible on its predecessors, allowing programs to
    store larger amounts of data in memory.
  \item
    x86-64 also provides 64-bit general purpose registers and numerous
    other enhancements. The original specification was created by AMD,
    and has been implemented by AMD, Intel, VIA, and others.
  \item
    It is fully backwards compatible with 16-bit and 32-bit x86 code.
  \item
    Examples are the Intel Core (i3/i5/i7) series and the AMD Athlon
    family.
  \end{itemize}
\item
  ARM:

  \begin{itemize}
  \itemsep1pt\parskip0pt\parsep0pt
  \item
    It is an instruction set architectures for processors based on RISC,
    developed by ARM Holdings.
  \item
    Thus ARM processors require significantly fewer transistors than
    typical x86 processors. Hence costs, heat and power use is reduced.
  \item
    Hence they are widely used in smartphones and tablets.
  \item
    A simpler design facilitates more efficient multi-core CPUs and
    higher core counts at lower cost, providing improved energy
    efficiency for servers.
  \item
    Popular ARM manufacturers include Qualcomm (the Snapdragon family of
    systems on a chip) and Apple (The A series). Apple was also the
    first company to release a 64-bit ARM chip on a consumer smartphone
    or tablet (the A7).
  \end{itemize}
\item
  SPARC:

  \begin{itemize}
  \itemsep1pt\parskip0pt\parsep0pt
  \item
    SPARC (from ``scalable processor architecture'') is a RISC
    instruction set architecture (ISA) developed by Sun Microsystems and
    introduced in mid-1987.
  \item
    Implementations of the original 32-bit SPARC architecture were
    initially designed and used in Sun's workstations and servers.
  \item
    The most recent iterations of the processor are Fujitsu's ``Venus''
    SPARC64 VIIIfx, used in the Japanese supercomputer ``K computer'',
    and the SPARC T5 introduced by Oracle in March 2013.
  \end{itemize}
\end{itemize}

\end{document}
